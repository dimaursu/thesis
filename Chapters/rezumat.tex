\section*{Rezumat}
Această teză acoperă proiectarea și implementarea unui sistem de  predare de la egal la egal.
Scopul principal este de a utiliza tehnologie de ultimă generație, în scopul de a crea o
rețea socială și de a explora tehnici neconvenționale de învățare virtuală.
Unul dintre obiectivele secundare ale sistemului este de a fi bazat doar pe tehnologii web standardizate,
în așa fel ca să fie accesibil, și nedependent de tehnologii proprietare.

Aplicația este publicată sub o licență pentru programe libere, în conformitate cu termenii
\href{http://www.gnu.org/licenses/agpl-3.0.html}{AGPLv3}. Codul sursă poate fi
descărcat de pe \href {https://gitorious.org/discite/discite/}{Gitorious}, și
utilizat fără restricții de către universități, licee sau persoane fizice.
Ruby și Ruby on Rails este utilizat ca fundație pentru proiect, completat de module în JavaScript.
Aplicația a fost dezvoltată și testată pe un sistem de operare GNU/Linux, dar aceasta nu este o dependență necesară,
fiind dezvoltată cît mai independent posibil de acest sistem particular. Alte
sisteme de operare asemănătoare cu Unix (MacOS, Solaris, FreeBSD), se presupune
că v-a rula programul foarte bine. Script-uri de deployment
și documentația sunt furnizate.

O instanță a aplicație Discite poate fi găsită la \url{http://discite.info}.
Numele aplicației, ``Discite", înseamnă ``a învăța" în latină.

\clearpage

