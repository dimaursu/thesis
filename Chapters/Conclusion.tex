\section*{Conclusions}
\phantomsection
As the thesis projects was developed, several technical challenges were met:
the functionality of the video chat had to be reduced, as sending tens
of video streams simoultaneuosly is currently impossible.
One of the reasons is that  decoding the video stream is a very computing
intensive operation. The connections using WebRTC are peer-to-peer so the server
is not a bottleneck, but users with older computers would not be able to handle
the decoding of many video streams. The network bandwidth imposes limitations,
and this was also taken into account when designing the application.
Also, managing the students and the chat would be difficult and disruptive for
the teacher.
A control panel to manage such use cases can be developed, but it's
out of the scope of this thesis.

The rating system that was planned initially as the feedback component in this
self-correcting system was not built, due to lack of time. In future versions
the rating system will be put at work, and it's efficienciency will be tested out.
The current system bets on self-organization, on students joining the platform
according to learning goals, prior knowledge and skills, and common interests.
The limitation that was introduced in the application is also subject to changes:
currently only the teacher can broadcast/share his screen and video/audio chat.
Another problem was with the presentations: HTTP, the protocol used for web page
delivery on the web, is a stateless protocol. Presentations have to be stateful,
so teacher and students will see the same slide at the same time, and the server
was used to achieve this, with Server-Sent Events, another new technology
standardised in HTML5. In the current iteration of the platform, only two events
are syncronized, '.next-slide' and '.prev-slide'. This poses issues when there is
a jump in page involved, or if some student joins later. A way of periodicaly checking
if all participants are at the same slide should be introduced.

Future work inlcudes the admin panel, a rewrite of the course page in NodeJS,
WebSockets as the sycronization channel, and a better course discovery experience.
Because of the lack of time and experience designing a big systems, the application
feels a bit half-backed and unstable, but it was a powerful learning experience.
Athough the web app was tested with Chrome and Firefox, the experience is still
bit flacky.
A lot of work was put in making PDF and ODF presentation uploads available: many
teachers already have their presentations in PDF, .odp or .ppt format, but building
this has failed, mainly because of weak skills in JavaScript and very complex libraries
used to render those files, like WebODF, PDF.js and ViewerJS; because of time contraints this
task was delayed.

The main objective has been reached: the application was build using modern
technologies such as WebRTC, SSE, it is interactive and engaging, much more than
traditional web pages. All this is possible with recent advances in the HTML
standardisation, and very efficient browser implementations.
\clearpage
