\section*{Introduction}
\phantomsection
``Discite'' is a web application for interactive peer-to-peer teaching.  The
idea is borrowed from dating sites, as one can pick a teacher or a student
based on his/her interests, ratings and rankings, skills one is looking for,
language of teaching, or other criterias. The peer-to-peer process of learning
allows screen sharing, video/voice broadcast and presentations. The teacher has
a tool to create courses at his disposal, and a tool for creating presentations
is planned in future versions.

The web application belongs to the MOOC domain (Massive Open Online Course),
which is a novel approach to distance education. MOOC is geared toward a large
number of participants, and is usually accesed via Web. In addition to
traditional e-learning course material, such as videos, readings, and problems,
MOOC's provide forums for students, teaching asistants, professors. Still, this
kind of interectivity is not on par with the real-life experience of learning,
and this may explain why the dropout rates are so high, which are about 90\% on
Coursera\citep{courseradropout}.

The interactivity that ``Discite'' brings is some of life proved teaching
techniques, where a student can interrupt and change the way a lesson goes.
Teacher and students can can interact by chat, video conference, doodling.  On
the down side, the teacher and students have to meet at a certain time. The
course recordings can be watched after that, but that wont bring the
aforementioned benefits.

While developing the Discite, several objectives were kept in mind. The idea was
to provide a web-based implementation of a new way to have peer-to-peer lessons,
using open standards and  based on cutting-edge technology. Some example of such
technology is the emerging standard WebRTC, wich enables realtime communication
directly in browser, without plugins of any kind. WebRTC is already implemented
in popular browsers, before the standardization process is finished. Another
cucial technology is SSE (Server sent events), which allows the server to send
messages to clients, in contrast to the traditional request-response cycle, where
only the client is the initiator of an action.

A unique feature is that every registered user can create courses, asthere is no
concept of a teacher at the application level, there are just course owners,
and anyone can see what teaching implies and spread their skills to a wide
audience. This feature is what makes the platform a peer-to-peer teaching one.
Peer learning has aspects of self-organization that are rarely present in
traditional pedagogical models of teaching and learning.  When a student is
struggling, having someone who is from the same generation as them helps to
create bridges in the learning gaps. A peer tutor can form examples and relate
to a student on an entirely different level than an adult educator.

A rating system should be built in order to avoid some of pitfalls in a peer to
peer learning system, notably students led astray by a misguided peer, or a
impolite, rude one. The rating system will act as the feedback component in this
self-correcting system, and although there is not concept of supervisor, one
could be created if real world examples show that the rating system is not
efficient enough. The current system bets on self-organization, on students
joining the platform according to learning goals, prior knowledge and skills,
and common interests.

This system is built with the hope that it will inspire students to learn and
to teach, and it will be met with the same enthusiasm as the first MOOC was meet,
as Jenny Mackness describes it, because back then in 2005, not only was the
course design unique, but so too was the learning experience.  Easy access to
advancing technologies means that learners can now take control of where, when,
how, what and with whom they learn. There has been a massive growth in online
social networking in recent years. The use of online and other web 2.0
technologies is becoming common. Increasingly some learners can, and do, choose
not to use the learning environment provided by a course or institution, but to
meet instead in locations of their choice, such as social networks, wikis or
blogs\citep{thefirstmoocexp}.

\clearpage
