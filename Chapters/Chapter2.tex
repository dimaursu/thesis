% Chapter 2

\chapter{Methods Used}
\label{Chapter2}
\lhead{Chapter 2. \emph{Methods Used}}

\section{Planning}
The application is estimated to reach the beta stage after a month of work, mid
February.
The most complicated part is expected to be the audio/video conference, as this
part of the application represents terra incongnito(a estimated 3 weeks work).
There are very few examples to rely on, and the draft is not yet finalized.
Also, big difficulties lay ahead with the presentation module: is quite hard to
make 20 presentation synchronize with the master one in real-time. It is
expected that this module will take about a week of work.

\section{Test-Driven Development}
Test-Driven Development is a software development methodology and strategy that
requires test to written first (the red stage), then write just enough code to
make the tests pass (the green stage), and the analyze the code written, and
change it if necessary (the refactor stage). By necessity we mean either changing
the code for performance reasons, readability, consistency with the rest of the
code base, or other reasons.

Rspec is the testing framework used for this project.
\subsection{Page-Objects}
Writing plain unit tests, or feature tests comes with one drawback: the tests
can be brittle(if the markup changes), and when a test fails, is usually hard to
tell what caused the fail to happen. Using page objects solves some of this
problem, as when the test fail, the error thrown points to the page that caused
the error.

\section{Software Requirements Specifications}

\subsection{Business requirements}
\begin{enumerate}
\item enables users registration
\item provides a way for users to create courses
\item enable interactive teaching process, featuring video presentation,
doodling
\end{enumerate}

\subsection{User requirements}
\begin{enumerate}
\item registration
\item choosing a course
\item the system should retain the user language preference
\item download materials posted by the teacher
\item chat with the teacher
\item rate the course and teacher
\end{enumerate}

\subsection{Functional requirements}
\begin{enumerate}
\item the application should be internationalised
\item file uploads via AJAX
\item the interface should be responsive (800x600 up to 4k resolution)
\item WebRTC video/audio streams
\end{enumerate}

\subsection{Quality-of-service requirements}
\begin{enumerate}
\item it should work without issues for at least 4 hours
\item each page should have tests
\item each module should be tested with rspec
\item the application should be behavior-driven tested with capybara + RSpec
\item supporting at least up to 200 users simultaneously (apprx. 10-20 courses
at a time)
\item deployment by a skilled administrator should take less than 30 min.
\end{enumerate}

\subsection{Implementation requirements}
\begin{enumerate}
\item the system should not require administrators (self-sustained)
\item there should be no differences between teachers and students outside of a
class
\item avoid loosing data during upgrading and downgrading
\item keep the application independent of the RDBMS solution
\end{enumerate}

\section{Components}
Rails 4 - It finally has support for streaming (with ActionController::Live),
and for server-sent events, that will be used to synchronize the presentation,
doodles.

Another major component will be a browser synchronizer, that will receive
from Rails, via server-sent events, which will dispatch messages to 4 objects
in the course view (Presentation, Media, Sketch and Chat)

\subsection{Storage}
For storing data, the application uses 2 methods: the persistent data is stored
in a SQL database, and the files uploaded by the users are stored in the
file system. For the purpose of archiving the courses there might be implemented
another storage function (for video, chat), but it will not be implemented in
version 1.0 of the application. Chats will be ephemeral, they will only be kept
in users browsers as long as the course is running.

\section{User Interface/User Experience}
Zurb Foundation was chosen as the basis of the application interface. It has
already implemented mos usual widgets, such as buttons, forms, headers, menu.
\subsection{Guidelines}
http://elementaryos.org/docs/human-interface-guidelines

\subsection{Course page}
The most important and complicated page in the whole system is the course page.
It will be a page split in four: the video chat, the text chat, the presentation
and the doodling area.

\section{Implementation}
The project was started by scaffolding a few rails models (course, user), and
integrating components such as Zurb Foundation, creating header/footer, front
page, a course creation form.

\subsection{i18n}
A selector will be provided in the header. On the user profile page this setting
can be chosen permanently. The URL should look like:
discite.info/[lang]/the/rest/of/the/path
