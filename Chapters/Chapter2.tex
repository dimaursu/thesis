% Chapter 2

\chapter{Introducing Discite}

\section{About}
``Discite'' is a web application for interactive peer-to-peer
teaching. As in mating sites, one can pick a teacher or a student based on
his/her interests, ratings and rankings, skills one is looking for, language of
teaching. The peer-to-peer process of learning allows screen sharing, video and
voice chat, presentations and doodling.

The web application belongs to the MOOC domain (Massive Open Online Course),
which is a novel approach to distance education.

\subsection{Objectives}
Provide an web-based implementation of a new way to have peer-to-peer lessons,
using open standards, based on cutting-edge technology.

Also, a useful feature is that not only teachers can create courses
(there is no concept of a teacher at the application level, there are just
course owners), and anyone can see what teaching implies, and spread their
skills to a wide audience.

The interactivity that ``Discite'' brings is some of life proved teaching
techniques, where a student can interrupt and change the way a lesson goes.
Teacher and students can alter the way a course goes, they can interact by chat,
doodling.
On the down side, the teacher and students have to meet at a certain time. The
course recordings can be watched after that, but that wont bring the
aforementioned benefits.


\section{User Interface/User Experience}
Zurb Foundation was chosen as the basis of the application interface. It has
already implemented mos usual widgets, such as buttons, forms, headers, menu.
\subsection{Guidelines}
http://elementaryos.org/docs/human-interface-guidelines

\subsection{Course page}
The most important and complicated page in the whole system is the course page.
It will be a page split in four: the video chat, the text chat, the presentation
and the doodling area.

As the time goes, more and more courses are expected to be taken online, and
this web application will fill a gap of teacher-student interaction through the
Internet.
\section{Features implemented}
\begin{itemize}
    \item pick a course based on your interests
    \item give the course and teacher a rating
    \item choose a preferred language
    \item filter by skills one is looking for
    \item screen sharing
    \item voice/video chat
    \item presentation
    \item doodling
\end{itemize}

As the thesis projects is developed, several technical challenges are expected:
in the screen sharing mode, decoding the video stream is a very computing
intensive operation. The connections using WebRTC are peer-to-peer, and  we
should not be worried about the server-side of the application but a user will
a older computer would not be able to handle the decoding of tens of video
streams. Also, the network bandwidth imposes limitations. This is why a
limitation must be introduced in the application: only the teacher can
broadcast/share his screen and video/audio chat. Presentations - HTTP, the
protocol used for web page delivery on the web, is a stateless protocol.
Presentations have to be stateful, so teacher and students will see the same
slide at the same time. Maintaining a video stream and audio chats should be
investigated: using the right audio codec this can be implemented performance
and bandwidth-wise, but theoretically such a feature can be abused, even
unintentionally. Having a few students interfering with each-other can disrupt
the lesson. A control panel to manage such use cases can be developed, but it's
out of the scope of this thesis.

