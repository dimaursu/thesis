% Chapter 1

\chapter{Introduction}
\label{Chapter1}
\lhead{Chapter 1. \emph{Introduction}}

\section{About}
``Discite'' is a web application for interactive peer-to-peer
teaching. As in mating sites, one can pick a teacher or a student based on
his/her interests, ratings and rankings, skills one is looking for, language of
teaching. The peer-to-peer process of learning allows screen sharing, video and
voice chat, presentations and doodling.

The web application belongs to the MOOC domain (Massive Open Online Course),
which is a novel approach to distance education.
The project is similar to several existing projects:
Coursera (\href{https://coursera.org}{\texttt{https://coursera.org}}),
Prezi (\href{https://prezi.com}{\texttt{https://prezi.com}}),
iversity (\href{htpps://iversity.org}{\texttt{https://iversity.org}}),
udacity (\href{https://udacity.com}{\texttt{https://udacity.com}})

What these systems are lacking is interactivity: most of them are just a site
hosting videos, tasks and a forum for students.

The interactivity that ``Discite'' brings is some of life proved teaching
techniques, where a student can interrupt and change the way a lesson goes.
Teacher and students can alter the way a course goes, they can interact by chat,
doodling.
On the down side, the teacher and students have to meet at a certain time. The
course recordings can be watched after that, but that wont bring the
aforementioned benefits.

As the time goes, more and more courses are expected to be taken online, and
this web application will fill a gap of teacher-student interaction through the
Internet.
Also, a useful feature is that not only teachers can create courses
(there is no concept of a teacher at the application level, there are just
course owners), and anyone can see what teaching implies, and spread their
skills to a wide audience.

\subsection{Objectives}
Provide an web-based implementation of a new way to have peer-to-peer lessons,
using open standards, based on cutting-edge technology.

\subsection{Methodologies}
Test Driven Development - is a approach to software development where tests are
written first, which fail, as no code can yet be run to fulfill the test
conditions. Then the minimal amount of code is written to make it pass, and then
a refactor of the code is done if needed.  The upside of this approach is that
the focus is kept on the necessary features. Also, as a refactor is done the
test suite immediately shows if those changes broke any functionality in other
parts of the application.  The downside: the test suite can get as big as actual
code that will be run. But the experience shows that it pays off in the long
term. The gains are: \citep{twelveBenefitsOfUnitTests}
\begin{itemize}
    \item Unit tests prove that your code actually works
    \item You get a low-level regression-test suite
    \item You can improve the design without breaking it
    \item It's more fun to code with them than without
    \item They demonstrate concrete progress
    \item Unit tests are a form of sample code
    \item It forces you to plan before you code
    \item It reduces the cost of bugs
    \item It's even better than code inspections
    \item It virtually eliminates coder's block
    \item Unit tests make better designs
    \item It's faster than writing code without tests
\end{itemize}

\section{Technologies and frameworks}
A brief list of technologies used:
\begin{itemize}
    \item Ruby - ``A dynamic, open source programming language with a focus on
        simplicity and productivity.
        It has an elegant syntax that is natural to read and easy to write.''
        \href{https://wwww.ruby-lang.org/en/}{\texttt{www.ruby-lang.org/en}}
    \item \href{http://rubyonrails.org}{\texttt{Ruby on Rails}} - a MVC
        (Model-View-Controller) server-side web development framework, backed by
        the Ruby language
    \item \href{http://foundation.zurb.com/}{\texttt{Zurb Foundation}} - A
        responsive front-end framework
    \item \href{https://github.com/plataformatec/devise}{\texttt{Devise}} for
        authentication
    \item \href{http://www.webrtc.org/}{\texttt{WebRTC}} Real-Time Communication
        implemented in browser, with a simple JavaScript API
\end{itemize}

\section{Why?}
These technologies are chosen because of previous experience with them, and they
are well tested and widely deployed on production environments.
The only new and untried technology is WebRTC: is a very young project, aiming
to enable creation of rich web applications with HTML5. It's still a
\href{http://dev.w3.org/2011/webrtc/editor/webrtc.html}{\texttt{draft}} at W3C
Nonetheless, the technology is already implemented in 2 popular browser: Google
Chrome and Mozilla Firefox.  Internet Explorer doesn't yet support it, but as
it's market share is below 10\% we can assume that more than 80\% of web users
\citep{browserStats} will be able to use the program. As the specification will
leave the draft stage, even more supported browsers are expected.

\section{Features implemented}
\begin{itemize}
    \item pick a course based on your interests
    \item give the course and teacher a rating
    \item choose a preferred language
    \item filter by skills one is looking for
    \item screen sharing
    \item voice/video chat
    \item presentation
    \item doodling
\end{itemize}

As the thesis projects is developed, several technical challenges are expected:
in the screen sharing mode, decoding the video stream is a very computing
intensive operation. The connections using WebRTC are peer-to-peer, and  we
should not be worried about the server-side of the application but a user will
a older computer would not be able to handle the decoding of tens of video
streams. Also, the network bandwidth imposes limitations. This is why a
limitation must be introduced in the application: only the teacher can
broadcast/share his screen and video/audio chat. Presentations - HTTP, the
protocol used for web page delivery on the web, is a stateless protocol.
Presentations have to be stateful, so teacher and students will see the same
slide at the same time. Maintaining a video stream and audio chats should be
investigated: using the right audio codec this can be implemented performance
and bandwidth-wise, but theoretically such a feature can be abused, even
unintentionally. Having a few students interfering with each-other can disrupt
the lesson. A control panel to manage such use cases can be developed, but it's
out of the scope of this thesis.

