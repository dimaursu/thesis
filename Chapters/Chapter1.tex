% Chapter 1
\chapter{The current state of online learning}

In recent years, online learning has seen a massive boom: thousands of people
enroll each in online classes, some of these being complementary to the
traditional courses, some being only available online. The second option has
spawn a whole new branch of e-learning, called Massive Open Online Courses, or
MOOC.

Bringing education online takes advantages of connectivity simmilar to those of
social networks,

The course can be taken by an immense number of stundents.
Also, when the privacy and anonymity is a concern, online courses provides obvious
advantages: discrimination based on age, social status, or other criterias is if
not impossible, easily avoided.
Online courses also give students flexibility in their learning
schedule, a highly customized *Time Tables*, and provides them with high quality
courses, that otherwise would be impossible to attend, due to geographical
location or financial reasons.
If a course is well prepared, little to no guidance is necessary from the teacher,
and the knowledge can <be preserved in time?>
Harward, edX, coursera, Khan academin.

The project is similar to several existing projects:
Coursera (\href{https://coursera.org}{\texttt{https://coursera.org}}),
Prezi (\href{https://prezi.com}{\texttt{https://prezi.com}}),
iversity (\href{htpps://iversity.org}{\texttt{https://iversity.org}}),
udacity (\href{https://udacity.com}{\texttt{https://udacity.com}})

What these systems are lacking is interactivity: most of them are just a site
hosting videos, tasks and a forum for students.

