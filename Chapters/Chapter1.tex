% Chapter 1

\chapter{Introduction}
\label{Chapter1}
\lhead{Chapter 1. \emph{Introduction}}

\section{About}
``Discite'' is a web application for interactive peer-to-peer
teaching. As in mating sites, one can pick a teacher or a student based on
his/her interests, ratings and rankings, skills one is looking for, language of
teaching. The peer-to-peer process of learning allows screen sharing, video and
voice chat, presentations and doodling.

The web application belongs to the MOOC domain (Massive Open Online Course),
which is a novel approach to distance education.
The project is similar to several existing projects:
Coursera (\href{https://coursera.org}{\texttt{https://coursera.org}}),
Prezi (\href{https://prezi.com}{\texttt{https://prezi.com}}),
iversity (\href{htpps://iversity.org}{\texttt{https://iversity.org}}),
udacity (\href{https://udacity.com}{\texttt{https://udacity.com}})

What these systems are lacking is interactivity: most of them are just a site
hosting videos, tasks, and a forum for students.

The interactivity that ``Discite'' brings is some of life proved teaching
techniques, where a student can interrupt and change the way a lesson goes.
Teacher and students can alter the way a course goes, they can interact by chat,
doodling.
On the down side, the teacher and students have to meet at a certain time. The
course recordings can be watched after that, but that wont bring the
aforementioned benefits.

I think that as the time goes, more and more of the courses will be taken
online, and my application will fill a gap of teacher-student interaction
through the Internet.
Also, a useful feature is that not only teachers can create courses
(there is no concept of a teacher at the application level, there are just
course
owners), and anyone can see what teaching implies, and spread their skills to a
wide audience.

\section{Planning}
Objectives:\\
Provide an web-based implementation of a new way to have peer-to-peer lessons,
using open standards, based on cutting-edge technology.
Methodologies\\
Test Driven Development - is a approach to software development where you write
tests first, which fail, as no code can yet be run to fulfill
the test conditions. Then we write the minimal amount of code to make it pass,
and then we refactor the code if needed.
The upside of this approach is that the project doesn't go wild: the focus is
kept on the necessary features. Also, as we refactor the application
the test suite immediately shows if our changes broke the functionality in other
parts of the application.
The downside: we write as much test code as actual code that will be run. But
the experience shows us that it pays off in the long term.\\
Technologies:
\begin{itemize}
    \item Ruby - ``A dynamic, open source programming language with a focus on
        simplicity and productivity.
        It has an elegant syntax that is natural to read and easy to write.''
        \href{https://wwww.ruby-lang.org/en/}{\texttt{www.ruby-lang.org/en}}
    \item \href{http://rubyonrails.org}{\texttt{Ruby on Rails}} - a MVC
        (Model-View-Controller) server-side web development framework, backed by
        the Ruby language
    \item \href{http://foundation.zurb.com/}{\texttt{Zurb Foundation}} - A
        responsive front-end framework
    \item \href{https://github.com/plataformatec/devise}{\texttt{Devise}} for
        authentication
    \item \href{http://www.webrtc.org/}{\texttt{WebRTC}} Real-Time Communication
        implemented in browser, with a simple JavaScript API
\end{itemize}
I choose these technologies as I already have experience with most of them. The
one that I'm not familiar with is WebRTC:
it's a very young project, aiming to enable creation of rich web applications
with HTML5. It's still a
\href{http://dev.w3.org/2011/webrtc/editor/webrtc.html}{\texttt{draft}} at W3C
Nonetheless, the technology is already implemented in 2 popular browser: Google
Chrome and Mozilla Firefox.
Features:
\begin{itemize}
    \item pick a course based on your interests
    \item give the course and teacher a rating
    \item choose a preferred language
    \item filter by skills one is looking for
    \item screen sharing
    \item voice/video chat
    \item presentation
    \item doodling
\end{itemize}
Technical challenges - screen sharing, audio/video chat is very
compute-intensive operation, and while the connections are peer-to-peer
a user will a older computer would not be able to handle the decoding of tens of
videos. This is why a limitation must be introduced in the
application: only the teacher can broadcast/share his screen and video/audio
chat. Presentations - HTTP, the protocol used for web page delivery
on the web, is a stateless protocol. Presentations have to be stateful, so
teacher and students will see the same slide at the same time.
Add a citing by \citet{tenderlove} tender love.
