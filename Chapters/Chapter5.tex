\newcolumntype{L}[1]{>{\raggedright\let\newline\\\arraybackslash\hspace{0pt}}m{#1}}
\newcolumntype{C}[1]{>{\centering\let\newline\\\arraybackslash\hspace{0pt}}m{#1}}
\newcolumntype{R}[1]{>{\raggedleft\let\newline\\\arraybackslash\hspace{0pt}}m{#1}}
\newcolumntype{d}[1]{D{.}{.}{#1} }
\section{Economic analysis}
\phantomsection

\subsection{Description}
Social networks are very popular these days, billions of users are using them
each day. Most of them are focused towards keeping in touch with your friends,
or making new ones. Discite purpose is to use the social nature of the web to
learn something new, or to teach other people a skill that you are good at. The
web as a learning platform has been exploited many times before, several
successful commercial projects being BlackBoard, Moodle.

\subsection{Planning}
Creating a schedule is one of the first steps of project management. Using a
Spiral Workflow model, the project development will go through 3 steps:
determining the requirements, writing tests and then develop enough to make the
tests pass, planning for the next iteration.

\subsection{Objectives}
Planning helps to figure out the most important parts of the project, which will
determine the user experience when using the software. Writing tests ahead, will
help keep the team on tracks. This technique, when tests are written first, and
the code after, is called Test Driven Development, and allows faster iteration
through ideas while giving developers an idea of where the project is heading.


\subsection{Work amount evaluation}
The whole development of Discite will consist of planning, development and
deployment. These 3 steps can be repeated several times, until the quality of
the project is considered optimal. The second step, development, is expected to
consume most of the resources, both in people involved, and hours, because it is
the most involved: beside implementing features, testing is necessary, bug
fixing, and constant changes, based on the feedback from the testing team.

\subsection{Schedule}
In order to have a better grasp of how much the project will take place,
extensive planning in the first iteration is necessary. This leads to a better
understanding of the project, and where the most difficult and time consuming
part are expected. With this information, the project can be divided in small
tasks and intermediate deadlines, thus assuring the project will have a logical
flow of development.
The duration of the project can be obtained from the formula:

\begin{eqnarray}
DA = FD - SD + RT, \label{4.1}
\end{eqnarray}
where DA stands for duration of action, FD denotes finish date, SD is start date
and RT means  reserve time.  Start and finish date is days when activities take
place and Reserve time is number of days exceed the initial estimate.

In the table \ref{BriefSchedule} we see a representation of the initial schedule
of the development process, it will use the following notations:
\begin{enumerate}
    \item[--] PM: Project manager
    \item[--] SD: Software developers
    \item[--] SA: System architect
    \item[--] PO: Product owner
    \item[--] PC: personal computer
    \item[--] D: Designer
\end{enumerate}
\newcommand{\specialcell}[2][c]{%
  \begin{tabular}[#1]{@{}c@{}}#2\end{tabular}}

\begin{table}[ht!]
	\centering
	\caption{Brief Schedule}
	{
		\renewcommand{\arraystretch}{1.25}
		\begin{tabular}{C{1cm}|C{4cm}|C{2cm}|C{2cm}|C{2cm}|C{4.5cm}}
		\hline
		Nr. & Activity name & Worker & Activity duration, days & Who approves the activity & Resources used \\
		\hline \hline
                1 & Define the concept & \specialcell{PM \\ SD \\ SA} & \multicolumn{1}{|c|}{\specialcell{11\\12\\14}} & PM, PO & Internet, PC, discussions with organization\\

                2 & UML and DB modelling & \specialcell{SA \\ SD} & \multicolumn{1}{|c|}{\specialcell{7\\7}} & PM & Internet, PC \\

                3 & Define technologies used & \specialcell{SD \\ SA} & \multicolumn{1}{|c|}{\specialcell{4\\2}} & PM & Internet, PC \\

                4 & Business logic & SD & \multicolumn{1}{|c|}{71} & PM & Internet, PC, IDE, books \\

                5 & Design &  D & \multicolumn{1}{|c|}{20} & PM, PO & Internet, PC, IDE, books \\

                6 & Testing and fixing & SD  &  \multicolumn{1}{|c|}{28} & PM, PO & Internet, PC, IDE, books \\
                7 & Deployment & SD & \multicolumn{1}{|c|}{9} & PM, PO & Internet, PC \\

                8 & Reports, follow-ups, future plans & PM & \multicolumn{1}{|c|}{4} & PO & Internet, PC \\
		 \hline
                 & TOTAL & & \multicolumn{1}{c|}{180} \\ \hline
		\end{tabular}
	}
\label{BriefSchedule}
\end{table}

According to the schedule presented above, the expected time of the development
is 180 days, plus 21 days reserve. Given this table we can count the days
dedicated by each person for this project.
\begin{enumerate}
    \item[--] Project Manager : 15 days
    \item[--] System Architect : 23 days
    \item[--] Software Developer : 131 days
    \item[--] Designer : 20 days
\end{enumerate}

\subsection{Economic proof}
With the help of a wide research space, it is possible to bring economical
proofs for the IT projects, basing on the specific of the concurrences in
economic relationships. In the conditions of a low degree of determination of
the marketing environment, of high prices volatility, small degree of prognoses
depth, a common business-plan can't forecast reliably the final results of the
business. With this in mind, one of the basic instruments is choosing the
methods, the right positions and indicators for the economical proofs. \\
In order to reach this goal conditions, we must do numerous scientifically
researches, subordinated to the primary goal and formulated by means of the
following objectives:
\begin{enumerate}
    \item[--] Studying the theoretical and methodical aspects of the business-planning in the conditions of the concurrency on the market;
    \item[--] Systematization, determining the methodology and specifying the indices for the economical proof of the business-plans in IT;
    \item[--] Study and analysis of the actual practice of economical proofs of the business-plans for IT in Republic of Moldova;
    \item[--] Developing methodological concepts of the proofs of the decision of investment in the conditions of risk and incertitude;
    \item[--]  Studying the evaluation criteria of the business-projects’ efficiency and elaboration of a mechanism of complex evaluation of these.
\end{enumerate}

\subsection{Material and non-material expenses}
The manufacture of products or goods require prime material. In general, these
materials fall into two categories. These categories are direct material and
indirect materials. Direct materials are also called productive and material
goods, raw materials, storage materials without descriptive purposes only.
Direct material for a developer uses their work computers and other devices such
as a flash drive, CD-R, and others. Examples of direct material for the
production of computers is plastic, aluminum and glass. The direct material for
a developer serves their working computer, laptops, digitizers and other devices
such as flash drive, CD, and others. In table 5.2 and 5.3 below are presented
the material and non material expenses that arise during its development.

\begin{table}[ht!]
	\centering
	\caption{Long term material actives}
	{
		\renewcommand{\arraystretch}{1.25}
		\begin{tabular}{C{1cm}|C{5.5cm}|d{1}|C{2cm}|d{2}}
		\hline
		Nr. & Name & Unit's price, MDL & Required quantity, units & Sum, lei\\
		\hline \hline
		1 & PC &  10404 &  3 & 31212 \\

		2 & Printer & 1623 & 1 & 1623 \\
	\hline
		\multicolumn{4}{c|}{TOTAL} & \multicolumn{1}{d{2}}{32835} \\
		\hline
		\end{tabular}
	}
\label{LongTermMaterial}
\end{table}


\begin{table}[ht!]
	\centering
	\caption{Long term non-material actives}
	{
		\renewcommand{\arraystretch}{1.25}
		\begin{tabular}{C{1cm}|C{4.5cm}|d{2}|C{2cm}|d{2}}
		\hline
		Nr. & Name & Unit's price, MDL & Required quantity, units & Sum, MDL\\
		\hline \hline
                1 & GNU/Linux OS & Free  &  3 & 0\\

		2 & Hosting & 480 & 5 & 2400 \\
		\hline
		\multicolumn{4}{c|}{TOTAL}& \multicolumn{1}{d{2}}{2400} \\
		\hline
		\end{tabular}
	}
\label{LongTermNonMaterial}
\end{table}

Direct expenses (Table \ref{DirectMaterialCosts}) will include logistics products that will be used for development cycle of the project.

\begin{table}[ht!]
	\centering
	\caption{Direct materials costs}
	{
		\renewcommand{\arraystretch}{1.25}
		\begin{tabular}{c|c|d{2}|c|d{2}}
		\hline
		Nr. & Name & Unit's price, MDL & Required quantity, units & Sum, MDL\\
		\hline \hline
		1 & Office paper &  52 &  2 & 104 \\

		2 & Printer ink & 150 & 2 & 130 \\

		3 & Pen & 10 & 15 & 150 \\

		4 & Marker & 20 & 6 & 120 \\

		5 & Whiteboard & 890 & 1 & 890 \\

		6 & Post-it notes & 5 & 8 & 40 \\
		\hline
		\multicolumn{4}{c|}{TOTAL} & \multicolumn{1}{d{2}}{1604} \\
		\hline
		\end{tabular}
	}
\label{DirectMaterialCosts}
\end{table}

\textbf{Conclusion:} In table \ref{LongTermNonMaterial} are indicated the
non-material expenses for this project. In table \ref{LongTermMaterial} are
described material expenses. The total expenses for material and non-material
actives used for the development of this project are 36839 MDL.

\subsubsection{Salary expenses}
For the job of the workers the salary is calculated according to the table 5.5,
where we can see the amount of work (time) done and the price for each kind of
job. As all the employees are working part-time jobs the number of hours worked
for each employee is less than the standard 40hours/week. The time spent by each
employee was tracked with OpenProject, a web-based project management system,
where each employee has to track the time spent on each task.


\begin{table}[ht!]
	\centering
	\caption{Salary expenses}
	{
		\renewcommand{\arraystretch}{1.25}
		\begin{tabular}{c|c|c|d{2}|d{2}}
		\hline
		Nr. & Position held & Am. of work, days & Individual salary, MDL/day & Salary fund, MDL\\
		\hline \hline
		1 & System architect &  23 & 640 & 14720.0\\

		2 & Project manager & 15 & 600 & 9000.0\\

		3 & Designer & 20 & 720 & 14400.0\\

		4 & Programmer & 131 & 700 & 91700.0\\
                \hline
                \multicolumn{4}{c|}{Total pay off for all the workers}& \multicolumn{1}{d{2}}{129820.0} \\
                \hline
                \multicolumn{4}{c|}{Social fund(23 \%)}& \multicolumn{1}{d{2}}{29858.6} \\
                \hline
                \multicolumn{4}{c|}{Medical assurance (4 \%)}& \multicolumn{1}{d{2}}{5192.8} \\
                \hline
                \multicolumn{4}{c|}{Totally work remuneration}& \multicolumn{1}{d{2}}{164871.4} \\
                \hline
		\end{tabular}
	}
\label{SalaryExpenses}
\end{table}

To make it more explicit, here it comes the formula in base of which was calculated:
\begin{eqnarray}
F_{rm} = 14720 + 9000 + 14400 + 91700 = 129820 (MDL) , \label{PayOff}
\end{eqnarray}
Where \(F_{rm}\) is “Fondul de Retribuire a Muncii”, and on its basis is calculated the \(FS\):

\begin{eqnarray}
FS =  F_{rm} \cdot C_{fs} , \label{FondulSocial}
\end{eqnarray}
where FS is the sum of the contributions for the Fondul Social (FS) and
\(C_{fs}\) is contribution quota for the state mandatory social assurance,
approved each year by the Law of Budget of state (in 2014 – 23\%).
\begin{eqnarray}
FS = 129820 \cdot 0.23 = 29858.6 (MDL),
\end{eqnarray}
and
\begin{eqnarray}
AM = F_{rm} \cdot C_{am},
\end{eqnarray}
where AM is Medical Assurance and \(C_{am}\) is medical assurance quota approved
each year by the Law of Budget for state medical assurance (in 2014 – 4\%), and
is being computed by
\begin{eqnarray}
AM = 129820 \cdot 0.04 = 5192.8 (MDL) ,
\end{eqnarray}
The sum \(F_{rm} + FS +AM\) will be the total expense for work retribution.
\begin{eqnarray}
Total = 129820 + 29858.6 + 5192.8 = 164871.4 (MDL)
\end{eqnarray}

\subsubsection{Indirect expenses}
The fixed means pay-off is the partial loss of the consumable properties and
value of the means during their usage, influenced by different factors and the
increase of the work productivity.
For computing electricity usage we’ll take in to account that PC uses 350W per
working hour (8 hours per day * 180 usage days, 1440 hours). So we’ll have:
\begin{eqnarray}
Total~ power~ usage =  \frac{ 3 \cdot 350 \cdot 1440 }{1000}  =  1512 kWh
\end{eqnarray}
\begin{table}[ht!]
	\centering
	\caption{Indirect expenses}
	{
		\renewcommand{\arraystretch}{1.25}
		\begin{tabular}{c|c|c|d{2}|d{2}}
		\hline
		Name & Measure & Quantity & Tariff, lei & Total, lei\\
		\hline \hline
		Energy consumed & KWh & 1512 & 1.58 & 2388.96 \\

		Internet services & Subscription/month &  3 & 160 & 480.0 \\

		Office rent & contract & 3 & 3500 & 10500.0 \\

		Office Meals & Subscription/month & 3 & 1500 & 4500.0 \\

		\hline
		\multicolumn{4}{c|}{Total}& \multicolumn{1}{d{2}}{17868.96} \\
		\hline
		\end{tabular}
	}
\label{IndirectExpenses}
\end{table}


\subsubsection{Wear calculation}
Table \ref{MaterialWearCost} shows the wear of equipment was used for the
execution of this project. Each equipment in the following table has the partial
loss of the consumable properties and value of the means during their usage,
influenced by different factors and the increase of the work productivity. The
formula by which we compute these data is
\begin{eqnarray}
FA = (V \cdot T)/T_{1},
\end{eqnarray}
where V is initial value of the active, \(T_{1}\)- denotes useful usage time of
the active and T is actual time the active will be used in project development

\begin{table}[ht!]
	%\centering
	\caption{Material wear cost}
	{
		\renewcommand{\arraystretch}{1.25}
                \begin{tabular}{c|d{2}|C{2cm}|d{2}|C{2cm}}
		\hline
                Long term mat. active & Init\ value, MDL & Useful usage time, d & Project duration, days & FA, MDL \\
		\hline \hline
                PC & 31212 &  \multicolumn{1}{c|}{2000} & 180 & 2809.0 \\

                Printer & 1623 & \multicolumn{1}{c|}{2000} & 180 & 146.07 \\
		\hline
		\multicolumn{3}{c|}{Total}& \multicolumn{1}{d{2}}{2955.07} \\
		\hline
		\end{tabular}
	}
\label{MaterialWearCost}
\end{table}

\subsection{Project Cost}
The total cost of the project may be computed by adding together all the
expenses, which is illustrated in table \ref{ProjectExpenses}.

\begin{table}[ht!]
	\centering
	\caption{Summary calculation related to the project}
	{
		\renewcommand{\arraystretch}{1.25}
		\begin{tabular}{c|d{2}|d{2}}
		\hline
		Computing articles & Value, MDL & \% \\
		\hline \hline
		Salary Expenses & 129820 &  69.9 \\

		Social fund & 29858.8 & 16.07 \\

		Medical assurance & 5192.8 & 2.8 \\

		Indirect expenses & 17868.96 & 9.62 \\

		Material expenses wear cost & 2955.07 & 1.59 \\
		\hline
		Total & 185695.43 & 100.0 \\
		\hline
		\end{tabular}
	}
\label{ProjectExpenses}
\end{table}


\subsection{Financial results}
The “Discite” project will have a subscription by year of 1500lei for
collocation with custom domain, and installation of a self-hosted version for
one single tax of 5000 lei. There are expected 60 subscriptions and 10
installation of the self hosted version in the first year.
Finally the Gross Income (CA) will be:


\begin{eqnarray}
CA_{B} = Q \cdot P = 1500 \cdot 100 + 40 \cdot 5000~MDL = 350000~MDL ~~(including~VAT)
\end{eqnarray}
The net income is computed by removing 20\% from the gross income:
\begin{eqnarray}
CA_{N} = CA_{B} - TVA = 350000 - 350000 \cdot 0.2 = 280000~MDL
\end{eqnarray}
The gross profit can now be computed following the formula:
\begin{eqnarray}
P_{B} = CA_{N} - C_{T} = 280000 - 185695.43 =  = 94304.57~MDL
\end{eqnarray}
Now that the gross profit is determined, we can now compute the net profit. To get the net profit we
have to subtract a certain percentage from the gross profit that is dependent on the type of entity (a person or a legal personality). In our case the company "NovaTech LLC" (Limited Liability Company) is a legal personality, and the percentage to subtract is 12\%. The indicators are as follows
\begin{eqnarray}
 I_{P} = P_{B} \cdot 0.12 = 94304.57~MDL \cdot 0.12 = 11316.54~MDL
\end{eqnarray}
Thus the net profit can be computed by subtracting the taxes from gross profit
\begin{eqnarray}
P_{N} = P_{B} - I_{P} = 94304.57~MDL - 11316.54~MDL = 82988.03~MDL
\end{eqnarray}
Some profitability indicators can be computed to see how well the product was sold
\begin{eqnarray}
Sales~ profitability = \frac{P_{B}}{CA_{B}} \cdot 100\% = \frac{94304.57}{350000} \cdot 100\% = 26.9\%
\end{eqnarray}
\begin{eqnarray}
Economic~profitability = \frac{P_{B}}{C_{T}} \cdot 100\% = \frac{94304.57}{185695.43} \cdot 100 = 50.7\%
\end{eqnarray}


\subsection{Economic Conclusion}
The project Discite seems to a be a profitable one, if the prognosis are met.
Given  the nature of the service provided, the sales can go worldwide at no
expenses or developer time. Further costs for development and maintenance are
low, and the subscription based business model makes the service profitable, and
convenient for consumers (mainly universities and schools), as they don't need
to maintain servers or handle upgrades. Also, the second option of having the
service (paying for installation) is necessary for customers that have special
requirements for privacy, or too slow internet connections.

\clearpage
