% Appendix
\section*{Appendix}
\phantomsection
Because the application is licensed unde AGPLv3, and the source code is available
for download, the application can be instaled on the hardware a university owns,
for example. Below is discussed how an administrator should proceed in order to
deploy the application and to maintain it fuctional.
\subsection{Installing the application}
Things to keep in mind:
Run every command with a RAILS\_ENV="something" as a prefix, if you want to
switch away from the default "development" environment that Rails uses. You should
deploy the application with the development profile only for debugging purposes,
use the production profile for all other uses.

Fist, check if you have ruby installed on your system: chances are that is already
there. Keep in mind that the program was written on rubinius-2.1.1, but with
small adjustments it will work fine on MRI, v2.0.
If ruby is not present (running irb in terminal doesn't do anything, for example,
you should install it by using your distribution recommended way, or by using
either ruby-install + chruby, or RVM. Both are specialized tools for compiling, installation
and switching to the proper ruby version.
The following system dependencies are needed:
\begin{enumerate}[topsep=5pt, partopsep=0pt,itemsep=3pt,parsep=1pt]
    \item[--] nodejs, or another javascript runtime;
    \item[--] graphicsmagick.
\end{enumerate}
Clone the repo (git is needed for this step:
\[ git \quad  clone \quad git@gitorious.org:discite/discite.git\]
Then, inside the directory, run:
\[ bundle \quad install \]
Before you deploy, it's a good idea to run the test suite
\[ rake \quad test:all\]
As a deployment tool, capistrano is used.
It supports the scripting and execution of arbitrary tasks, and includes a set of sane-default deployment workflows.
configure the relevant fields in \(config/deploy\), then run
\[ cap \quad production \quad deploy\]
\subsection{Database Configuration}
Edit the \(config/database.yml\) file, and run:
\[rake \quad db:create\]
Database initialization
\[rake \quad db:migrate\]
If you are doing development, you may want to run the database seeder script:
\[rake \quad  db:seed\]
\subsection{Server setup}
Now you should be ready to run
\[./bin/rails server \quad -p \quad  <port>\]

Some of the tools that are used in development are rubocop, a static code analyzer,
the standard ruby debugger, and simplecov, to have feedback regarding test coverage.
All of these tools are available from the standard distribution channel, namely RubyGems.


